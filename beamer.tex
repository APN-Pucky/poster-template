%%% beispiel.tex --- 

%% Author: s_kaul01@PRIS
%% Version: $Id: beispiel.tex,v 0.0 2010/02/10 14:23:59 s_kaul01 Exp$

\documentclass{beamer}

\usepackage[pantone312,english]{wwustyle}

\usepackage[ngerman]{babel}
\usepackage[utf8]{inputenc}
\usepackage[T1]{fontenc}


%%\revision$Header: /u/s_kaul01/tmp/beispiel.tex,v 0.0 2010/02/10 14:23:59 s_kaul01 Exp$

\author{Max Mustermann}
\title{Zur Benutzung der WWU-Beamer Vorlage!}
\subtitle{Allgemeine Hinweise und Benutzung der WWU-Vorlage}

\begin{document}

\begin{frame}[plain]
  \maketitle
\end{frame}

\begin{frame}
  \frametitle{Folientitel sollten kurz und pr\"agnant sein}
  \begin{block}{Hervorhebungen}
    \textbf{Wenn man Dinge hervorheben m\"ochte nutzt man entweder Fettdruck,}
    \textit{ kursive Schrift} \alert{ oder das Schl\"usselwort ``alert''}.
  Auch ``itemize''-Umgebungen werden von der Stilvorlage überschrieben:
  \end{block}
  \pause
  \begin{itemize}
    \item So wird sichergestellt,
    \item dass alle Elemente der Präsentation 
    \item dieselbe Farbe nutzen.
  \end{itemize}
\end{frame}

\begin{frame}
  \frametitle{Ein Alerted-Block}
  \begin{alertblock}{Achtung!}
    Hier kommt Rot ins Spiel!
  \end{alertblock}
\end{frame}

\begin{frame}
  \frametitle{Ein Example-Block}
  \begin{exampleblock}{Beispiel}
    Hier kommt Grün ins Spiel!
  \end{exampleblock}
\end{frame}

\end{document}
