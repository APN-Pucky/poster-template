\documentclass{beamer}

\usepackage[pantone312,english]{wwustyle}

\usepackage[ngerman]{babel}
\usepackage[utf8]{inputenc}
\usepackage[T1]{fontenc}

% Uncomment the following two lines if you want to prepare your document for
% the fast mode.
% \usetikzlibrary{external}
% \tikzexternalize

\author{Max Mustermann}
\title{Zur Benutzung der WWU-Beamer Vorlage!}
%\institutelogo{Logo on title frame}
%\institutelogosmall{Logo on other frames}
\subtitle{Allgemeine Hinweise und Benutzung der WWU-Vorlage}

\begin{document}

%%%%%%%%%%%%%%% WWUstyle "fast" mode %%%%%%%%%%%%%%%%%%%%%
% Do the following steps in order to speed up the compilation time of your
% presentation:
%
% 1. Include the externalization tikz library in the preamble of your document.
%    This is always recommended if you are using tikz in your document.
% 2. Uncomment the \wwupreparefastmode command below
% 3. Compile your document with command line option '-shell-escape',
%    e.g.: 'pdflatex -shell-escape beamer.tex'
% 4. Comment (or delete) the \wwupreparefastmode
% 5. Add option 'fast' to the 'wwustyle' package declaration line.
% 6. Be happy!

% \wwupreparefastmode


\begin{frame}[plain]
  \maketitle
\end{frame}

\begin{frame}
  \frametitle{Folientitel sollten kurz und pr\"agnant sein}
  \begin{block}{Hervorhebungen}
    \textbf{Wenn man Dinge hervorheben m\"ochte nutzt man entweder Fettdruck,}
    \textit{ kursive Schrift} \alert{ oder das Schl\"usselwort ``alert''}.
  Auch ``itemize''-Umgebungen werden von der Stilvorlage überschrieben:
  \end{block}
  \pause
  \begin{itemize}
    \item So wird sichergestellt,
    \item dass alle Elemente der Präsentation 
    \item dieselbe Farbe nutzen.
  \end{itemize}
\end{frame}

\begin{frame}
  \frametitle{Ein Alerted-Block}
  \begin{alertblock}{Achtung!}
    Hier kommt Rot ins Spiel!
  \end{alertblock}
\end{frame}

\begin{frame}
  \frametitle{Ein Example-Block}
  \begin{exampleblock}{Beispiel}
    Hier kommt Grün ins Spiel!
  \end{exampleblock}
\end{frame}

\end{document}
